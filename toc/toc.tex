\documentclass[12pt,a4paper]{article}
\setcounter{secnumdepth}{0}
\usepackage{gensymb}
\usepackage{amsmath}
\usepackage{amssymb}
\usepackage{enumitem}
\usepackage{graphicx}
\usepackage{sansmath}
\usepackage{pst-eucl}
\usepackage[UKenglish]{isodate}
\usepackage[UKenglish]{babel}
\usepackage{float}
\usepackage[numbered,framed]{matlab-prettifier}
\usepackage[T1]{fontenc}
\usepackage{setspace}
\usepackage{sectsty}
\usepackage[colorlinks=true,linkcolor=blue,urlcolor=black,bookmarksopen=true]{hyperref}
\renewcommand{\baselinestretch}{2.0}
\usepackage[margin=0.1in]{geometry}
\title{Table of contents of files for the undergraduate research scholarship programme}
\author{Brenton Horne}

\begin{document}
\maketitle

\begin{tabular}{|m{15em}|m{30em}|}
    \hline
    File & Description \\\hline
    computeTrajectory/back.py & Runs back-simulation for Solar system (Venus-Jupiter and Sun) and specified asteroid number. Saves the output from the simulation as CSV files.\\\hline
    computeTrajectory/base.py & Runs forward simulation on just the Sun and the planets from Venus to Neptune. Saves the output from the simulation as CSV files.\\\hline
    computeTrajectory/close.py & Runs simulation on close encounters mentioned in closeEncounters.csv between asteroids and the planets. Saves the output from the simulation as CSV files.\\\hline
    computeTrajectory/main.py & Runs forward simulation for Solar system (Venus-Jupiter and Sun) and specified asteroid number. Saves the output from the simulation as CSV files.\\\hline
    eqns/eqns.pdf & Equations (1) and (2) from Lowry et al. (2020), this document was used to generate a picture for the presentation. \\\hline
    plot/back.py & Plot $y$ against $x$ (in au units) for the planets from Venus to Jupiter and the specified asteroid's first clone (the one with no positional uncertainty accounted for). Then plots semi-major axis in au units, eccentricity and inclination against time for the specified asteroid's first clone. \\\hline
    plot/backAngles.py & Generates perihelion longitude and nodal longitude against time plots for back-simulation. \\\hline
    plot/belt.py & Generates the asteroid belt plot in the presentation and saves to a PNG file. \\\hline
    plot/HmagvsSMA.jl & Writes Hmag vs SMA data for each asteroid in the Svea family (specified by 416\_svea.tab) to HmagvsSMA.csv, which will then be used by HmagvsSMA.py to generate the Hmag vs SMA plot shown in the slideshow. \\\hline
    \end{tabular}

\begin{tabular}{|m{15em}|m{30em}|}
    \hline
    File & Description \\\hline
    plot/HmagvsSMA.py & Plots Hmag vs SMA for each asteroid on the same plot window and saves to a SVG file, which it then converts to a PNG file using ImageMagick. \\\hline
    plot/main.py & Plots ($y$ in au versus $x$ in au) the planets from Venus to Jupiter with every asteroid clone (one asteroid clone per plot) and saves to a SVG file and then converts the SVG to a PNG file using ImageMagick. Plots semi-major axis (au units) versus time for each asteroid clone (one asteroid clone per plot) and saves the plot as a SVG file and then converts it to PNG using ImageMagick. Plots eccentricity versus time for each asteroid clone (one asteroid clone per plot) and saves the plot as a SVG file and then converts it to PNG using ImageMagick. Plots inclination versus time for each asteroid clone (one asteroid clone per plot) and saves the plot as a SVG file and then converts it to PNG using ImageMagick.\\\hline
    plot/NEA.py & Scatter plots final eccentricity values vs final semi-major axis (SMA) values for each asteroid's 0th clone (no positional uncertainty considered) and colour grades each point by difference in eccentricity or SMA over the trajectory. Saves the plots as SVG files and then converts it to PNG files using ImageMagick.\\\hline
    plot/withEarth.py & Plots each asteroid clone (one asteroid clone per window) with the Earth, was used to determine which asteroids were just Mars crossers but otherwise stayed out of the inner Solar system. Saves the plots as SVG files and then converts it to PNG files using ImageMagick.\\\hline
\end{tabular}

\begin{tabular}{|m{15em}|m{30em}|}
    \hline
    File & Description \\\hline
    plot/plotAll.py & Plots the planets from Venus to Mars and each asteroid clone (one clone per plot) and saves the plot as a SVG file and converts this to a PNG file using ImageMagick. Plots Venus and Earth and each asteroid clone (one clone per plot) and saves the plot as a SVG file and converts this to a PNG file using ImageMagick. Plots Venus and each asteroid clone (one clone per plot) and saves the plot as a SVG file and converts this to a PNG file using ImageMagick. Plots the planets from Venus to Jupiter and each asteroid clone (one clone per plot) and saves the plot as a SVG file and converts this to a PNG file using ImageMagick. Was originally kept in the same directories that the CSV files for the simulations was.\\\hline
    plot/quickPlot.py & Plots 0th clone (no positional uncertainty accounted for) for specified asteroid number plus all the planets in the simulation.\\\hline
    plot/whenEsc.py & Prints the escape time for each planet and each asteroid clone that escapes. \\\hline
    report/report.pdf & Short report for the programme. \\\hline
    save/asteroid.py & Saves every asteroid clone along with all the planets from Venus to Neptune and the Sun to a .bin file (one asteroid with all its clones per .bin file). \\\hline
    save/base.py & Save planets from Venus to Neptune and the Sun to a .bin file. \\\hline
    save/closeEncounter.py & Save planets from Venus to Neptune, the Sun and each of the asteroid clones that come within 5 Hill radii of the planets to .bin files (one .bin file per asteroid clone). \\\hline
    toc/toc.pdf & The table of contents for the files in this tarball.\\\hline
    .bashrc & ~/.bashrc file used on the HPC, complete with functions and aliases that make working on the HPC easier. \\\hline
\end{tabular}

\begin{tabular}{|m{15em}|m{30em}|}
    \hline
    File & Description \\\hline
    execute.pbs & PBS script that executes a computeTrajectory script across multiple different cores of the HPC. \\\hline
    findCloseEncounters.jl & Prints close encounters (distance < 5 Hill radius) between planets and asteroid clones in CSV format with time ranges in which the encounter occurs and the asteroid number and clone number. \\\hline
    journalArticle/article.pdf & Journal article I wrote up for this project. \\\hline
    presentation/presentation.pdf & A PDF copy of the presentation I delivered on 12 January 2022 at the first annual astromeeting. \\\hline
\end{tabular}
\end{document}