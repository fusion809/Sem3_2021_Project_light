\documentclass[12pt,a4paper]{article}
\setcounter{secnumdepth}{0}
\usepackage{gensymb}
\usepackage{amsmath}
\usepackage{amssymb}
\usepackage{enumitem}
\usepackage{graphicx}
\usepackage{sansmath}
\usepackage{pst-eucl}
\usepackage[UKenglish]{isodate}
\usepackage[UKenglish]{babel}
\usepackage{float}
\usepackage[numbered,framed]{matlab-prettifier}
\usepackage[T1]{fontenc}
\usepackage{setspace}
\usepackage{sectsty}
\usepackage[colorlinks=true,linkcolor=blue,urlcolor=black,bookmarksopen=true]{hyperref}
\renewcommand{\baselinestretch}{2.0}
\usepackage[margin=0.1in]{geometry}
\title{Table of contents of files for the undergraduate research scholarship programme}
\author{Brenton Horne}

\begin{document}
\maketitle

\begin{tabular}{|m{15em}|m{30em}|}
    \hline
    File & Description \\\hline
    computeTrajectory/back.py & Runs back-simulation for Solar system (Venus-Jupiter and Sun) and specified asteroid number. \\\hline
    computeTrajectory/base.py & Runs forward simulation on just the Sun and the planets from Venus to Neptune. \\\hline
    computeTrajectory/close.py & Runs simulation on close encounters mentioned in closeEncounters.csv between asteroids and the planets. \\\hline
    computeTrajectory/main.py & Runs forward simulation for Solar system (Venus-Jupiter and Sun) and specified asteroid number. \\\hline
    eqns/eqns.pdf & Equations (1) and (2) from Lowry et al. (2020), this document was used to generate a picture for the presentation. \\\hline
    plot/back.py & Plot $y$ against $x$ (in au units) for the planets from Venus to Jupiter and the specified asteroid's first clone (the one with no positional uncertainty accounted for). Then plots semi-major axis in au units, eccentricity and inclination against time for the specified asteroid's first clone. \\\hline
    plot/backAngles.py & Generates perihelion longitude and nodal longitude against time plots for back-simulation. \\\hline
    plot/belt.py & Generates the asteroid belt plot in the presentation. \\\hline
    plot/HmagvsSMA.jl & Writes Hmag vs SMA data to HmagvsSMA.csv, which will then be used by HmagvsSMA.py to generate the Hmag vs SMA plot shown in the slideshow. 
\end{tabular}
\end{document}