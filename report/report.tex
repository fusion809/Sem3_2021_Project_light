\documentclass[12pt,a4paper,openright]{article}
\setcounter{secnumdepth}{0}
\usepackage{gensymb}
\usepackage{amsmath}
\usepackage{amssymb}
\usepackage{enumitem}
\usepackage{graphicx}
\usepackage{sansmath}
\usepackage{pst-eucl}
\usepackage[UKenglish]{isodate}
\usepackage[UKenglish]{babel}
\usepackage{float}
\usepackage[numbered,framed]{matlab-prettifier}
\usepackage[T1]{fontenc}
\usepackage{setspace}
\usepackage{sectsty}
\usepackage[colorlinks=true,linkcolor=blue,urlcolor=black,bookmarksopen=true]{hyperref}
\renewcommand{\baselinestretch}{2.0}
\usepackage[margin=0.1in]{geometry}
\title{Short report on scholarship programme}
\author{Brenton Horne}

\begin{document}
\maketitle

\section{What I did}
For the Health, Engineering and Sciences undergraduate research scholarship programme (HES-URSP), I, under the supervision of Dr Timothy Holt, studied the Svea family of asteroids. My goals were to determine the future trajectory and the origins of the family, and also a little about the family's present state. 

This family consists of 48 members, with each member being primarily located within the inner asteroid belt, which is between the orbits of Mars and Jupiter. The eponymous member of the family is 329 Svea, an asteroid that is approximately 80 kilometres in diameter, roughly nine times larger than Mount Everest. The only other named member of the family is 13977 Frisch, which is roughly 9 kilometres in diameter, still larger than Mount Everest (which is 8,848.6 metres tall). I also learnt that the family seems to be mostly carbonaceous, that is, primarily composed of carbon-based materials like graphite. Asteroid families like the Svea family are believed to originate from the breakup of larger asteroids due to a collision. 

I discovered that the family likely originated 269-376 million years ago and that the members of the family are on diverse trajectories. Some were on trajectories that kept them entirely within the asteroid belt; others were on trajectories that could lead to them entering the Kirkwood gap caused by 3:1 resonance with Jupiter and being ejected from the Solar system; others occasionally crossed into the orbit of Mars, but otherwise stayed within the asteroid belt; others crossed into the orbits of Mars and Earth but otherwise stayed within the asteroid belt; and others still crossed into the entire inner Solar system, crossing even into Venus and Mercury's orbits. 

\section{What skills I developed}
During this project I learnt about the Yarkovsky effect, the effect by which thermal radiation being emitted by an asteroid can alter its trajectory. I learnt a little about Keplerian orbital elements, which are various parameters that describe the orbits of celestial bodies. I learnt about asteroid families and the various populations of small bodies in the Solar system. I also learnt two techniques for determining the age of a family. One used numerical back-simulation and the point of convergence of the angles of longitude of ascending node ($\Omega$) and argument of periapsis ($\omega$) to determine age, if age was less than 100 million years. The other used the amount of drift due to the Yarkovsky effect to determine age, which was especially useful if the family was older than 100 million years, which it was in this case. 

I also learnt how to write PBS scripts to parallelize jobs on USQ's High Performance Computing (HPC) facilities, as predicting the future (100 million year) trajectory of the family required this. I also learnt how to predict the motion of asteroids using the REBOUND Python library. I also learnt: how to produce publication-quality plots using Python's Matplotlib library; how to connect to USQ's student VPN, which was required in order to access the HPC; how to connect to the HPC using SSH; how to transfer files between the HPC and my local machine using SSH; how to write a report in Overleaf, which is where I wrote a journal article detailing everything I did; how to add collaborators to a private repository on GitHub, which I did so my supervisor could view the code I was working on; the value of leaving what I am working on for a while and coming back to it later, when I know there is an issue with my work that I just cannot identify; and that modelling asteroids in the Solar system in the real-world usually involves disregarding relativistic effects.

I also learnt a little about the difference in time between Queensland and the US state of Colorado, where my supervisor lives, and hence a little about the challenges that can exist when working with an international team of collaborators.

Additionally, I learnt that I seem to have a knack for technical computing, as I was able to finish much of the coding much earlier than my supervisor anticipating. In fact, my supervisor offered to write me a recommendation for future jobs involving technical computing for astronomy projects. So this is something else I got out of this project, a connection to someone that could help me get a job in the future. My supervisor also offered to get me in touch with someone that could come up with an honours project for me that involves general relativity. 

This project also taught me that while there are definitely aspects of research I dislike, like reading often very long and cryptically-worded journal articles, I do generally enjoy it. I like the rush that I get from solving a challenging a problem, like getting a computer program I have written to work and I like the feeling of being potentially the first person ever to learn something my research has revealed. So I know that doing an honours year and then hopefully a PhD is the right choice for me. Which is good to know as it is my intention to become a lecturer in applied mathematics, statistics, computing and maybe even physics. 

\section{Other aspects of my experience with the project}
When I originally started on this project, I feared that I may have bitten off more than I could chew by asking for the 10 week project. The reason why was that I had not done any astronomy or physics subjects at USQ and this was over my much-needed holidays, so I thought maybe I would end up at week 7 into the project hating the project and regretting that I had selected the 10 week project instead of the 6 week project. The reason I selected the 10 week project in the first place was mostly because it involved working with the HPC and I thought as an aspiring mathematician that would be helpful experience for me. 

Turns out these fears were unfounded, everything physics and astronomy related that I did not already know I was able to pick up as required during the project. As for the fear that I would end up hating the project, this was unfounded too, as I found the project easier than anticipated and it was often good to have something productive to do over my holidays. One of the subjects I took in semester 1, 2021 actually greatly helped to prepare me for this project, MAT2409 (High Performance Numerical Computing). In it, I learnt about the art of report-writing and how to write good quality Python code, both of which were essential for this project. I would recommend listing MAT2409 as a recommended prerequisite for this project in the future. 

Instead of working three days a week, I split my workload over the entire week, although I will confess I do not believe I spent an average of 24 hours a week for all ten weeks on this project, I suspect I managed to get away with less than that due to my skill with programming and report writing and past experience with LaTeX, Linux and Python. Part of how I managed to keep ahead on this project, despite likely not spending quite as much time on it as anticipated would be necessary was that whenever my supervisor gave me a job to do, I would not relax until I had done it, or at least done all I could do of it (which was sometimes the case for when what I needed to do involved running a simulation that may take several hours or even days to complete). It just turned out that I needed less time to get everything he asked me to do done than he may have anticipated. 

\end{document}